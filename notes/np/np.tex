\documentclass[]{article}
\usepackage[utf8]{inputenc}
\usepackage[portuges]{babel}
\usepackage[a4paper,margin=2.5cm]{geometry}
\usepackage{icomma, amsmath, multirow, graphicx}
\title{R{\o}mer no stellarium}
\author{}

\begin{document}
\maketitle

\abstract
\noindent
Tentou-se determinar o valor da velocidade da luz com diferentes métodos
inspirados no de R{\o}mer usando ``observações'' feitas no stellarium. Os
resultados parecem ser dececionantemente imprecisos.


\section{Introdução}
Seja $\tilde t$ o instante em que se dá uma ocultação ou emersão de Io na sombra
de Júpiter e $t$ o instante em que ela é observada na Terra. A relação entre
estes dois instantes é
\begin{equation}
  t=\tilde t+\frac{\tilde d}{c},
\end{equation}
onde $c$ é a velocidade da luz e $\tilde d$ é a distância entre a posição de Io no
momento da ocorrência e a posição da Terra no momento da observação. Por razões
de simplicidade, é feita a aproximação de substituir esta distância pela que
separa a Terra de Júpiter no momento da observação. O erro cometido terá, em
termos relativos, a ordem de grandeza da proporção entre o raio orbital de Io e
a distância Terra-Júpiter, cujo valor máximo ronda os $7,5\times10^{-4}$.

Faz-se a suposição de que o intervalo entre duas ocultações ou emersões
sucessivas de Io tem duração muito aproximadamente constante $\tilde T$ que será
identificada com o seu período orbital médio.  (Atenção, refere-se aqui o
\emph{intervalo entre os eventos,} não o intervalo entre as suas
\emph{observações} na Terra). Então, a duração do intervalo de tempo entre a
observação de duas ocultações ou duas emersões de Io na mesma fase de
aproximação ou regressão da Terra a Júpiter, entre as quais Io complete $n$
revoluções, é então
\begin{equation}\label{eq:aa}
  t_n-t_0=\tilde t_n-\tilde t_0 + \frac{d_n-d_0}{c}=n\tilde T+
  \frac{d_n-d_0}{c}.
\end{equation}

\section{Métodos usados}
\subsection{Método simples}
Considerando um quarteto de eventos formado por dois pares $(a)$ e $(b)$ de duas
emersões, de duas oculta\-ções, ou de duas emersões e duas ocultações, mas
escolhidos os dois pares de forma a que o número de revoluções de Io entre as
duas ocorrências de cada par seja iguais, podemos eliminar o termo em $\tilde T$
na eq.~\eqref{eq:aa}, resultando para $c$
\begin{equation}\label{eq:bb}
  c=\frac%
  {\left(d^{(a)}_n-d^{(a)}_0\right)-\left(d^{(b)}_n-d^{(b)}_0\right)}%
  {\left(t^{(a)}_n-t^{(a)}_0\right)-\left(t^{(b)}_n-t^{(b)}_0\right)}.
\end{equation}
 É de esperar que se o quarteto de eventos for formado por dois pares
de emersões ou dois pares de ocultações, ocorra um quase cancelamento da
diferença no denominador, o que resulta numa amplificação de erros do numerador.
Por isso mesmo, este método será usado considerando sempre um quarteto de
eventos formado por um par de emersões e um par de ocultações.%
\footnote{Este método, adotando pares de emersões e ocultações próximas das
  conjunções e oposições de Júpiter, é frequentemente apresentado como tendo
  sido o usado por R{\o}mer na sua comunicação à Academie Royale des Sciences em
1676.}

\subsection{Regressão com pares}
Podemos reescrever a eq.~\eqref{eq:aa} como
\begin{equation}
  \frac{t_n-t_0}{n}=\tilde T+\frac{1}{c}\frac{d_n-d_0}{n}.
\end{equation}
Considerando vários pares de emersões ou de ocultações definindo intervalos com
durações (ou seja, números de revoluções de Io) arbitrários e construindo um
gráfico cujas abcissas e ordenadas correspondam, respetivamente, aos valores de
$(d_n-d_0)/n$ e de $(t_n-t_0)/n$ de cada intervalo, o declive da reta que
melhor se ajusta aos diferentes pontos é uma estimativa do inverso de $c$, e a
sua ordenada na origem do período orbital de Io.

\subsection{Regressão com quartetos}
Rescrevemos a eq.~\eqref{eq:bb} na forma
\begin{equation}
  \left[
    {\left(t^{(a)}_n-t^{(a)}_0\right)-\left(t^{(b)}_n-t^{(b)}_0\right)}
  \right]=
  \frac{1}{c}
  \left[
    {\left(d^{(a)}_n-d^{(a)}_0\right)-\left(d^{(b)}_n-d^{(b)}_0\right)}
  \right]
\end{equation}
O declive da reta que melhor se ajusta aos pontos de um gráfico cujas abcissas e
ordenadas sejam os valores das diferenças entre parêntesis retos é uma
estimativa da velocidade da luz (ou do seu inverso, consoante o que escolhermos
como abcissas ou ordenadas.

\subsection{Batota!}
A ideia é usar o stellarium para determinar as distâncias e os tempos a usar nas
expressões dos métodos descritos acima. Mas o stellarium permite ainda uma outra
possibilidade, que consiste em observar o mesmo evento (ocultação ou emersão de
Io) visto da Terra e de Júpiter e estimar $c$ considerando a distância
Terra-Júpiter e a diferença temporal entre os instantes de observação do evento
nos dois locais.

\section{Determinação dos eventos com o stellarium}
Usando um script stellarium, foram geradas curvas de magnitude como função do
tempo para todas as ocultações e emersões de Io durante o ano sinódico de
Júpiter de 2024-12-08 a 2026-01-08. O momento central de cada evento e a
distância a Júpiter nesse instante foram estimados por interpolação. Estes
valores estão disponíveis para todas as emersões e ocultações de Io no ano
sinódico referido.

Tentou-se repetir os cálculos para outros anos sinódicos mas, considerando
apenas eventos extremos (mais próximos das oposições e das ocultações)
aplicou-se um procedimento mais rápido em que apenas se fez uma
aproximação ao segundo mais próximo da magnitude central de cada evento. As
datas dos eventos foram determinadas para o período de 2020-07-14 a 2024-12-07,
mas as horas e, ainda menos, as distâncias a Júpiter não foram determinadas para
a grande maioria das ocorrências.


\section{Resultados obtidos}
\subsection{Batota!}
Na Tabela~\ref{tab:aa} apresentam-se os dados relativos a duas emersões e duas
ocultações escolhidas arbitrariamente, e os valores das estimativas resultantes
para a velocidade da luz.  Os reultados obtidos são bastante próximos do valor
definido no SI.
\begin{table}[htb]
  \begin{center}
    \begin{tabular}{cccccc}
      \hline
      Evento & $t_T$ & $t_J$ & $\delta t$ (s) & $\overline{TJ}$ (UA) & $c$ (m/s) \\
      \hline
      Emersão  & 2021-10-30 20:31:10 & 2021:10:30 19:52:25 & 2325 & 4,65728 & $2,9966\times10^8$\\
      Ocultação& 2023-08-14 20:54:05 & 2023-08-14 20:14:26 & 2379 & 4,73848 & $2,9797\times10^8$\\
      Emersão  & 2025-03-24 13:47:06 & 2025-03-24 13:02:36 & 2670 & 5,35176 & $2,9985\times10^8$\\
      Ocultação& 2025-09-08 14:53:45 & 2025-09-08 14:06:33 & 2832 & 5,64948 & $2,9843\times10^8$\\
      \hline
    \end{tabular}
  \end{center}
  \caption{\label{tab:aa}Momento da observação de emersões e ocultações de Io na
  Terra e em Júpiter (UTC), diferença temporal, distância Terra-Júpiter no
momento da observação na Terra (em UA) e estimativa de $c$.}
\end{table}
\subsection{Método simples}
Em cada ano
sinódico de Júpiter, consideraram-se quartetos de eventos extremos, isto é, o
mais próximos possíveis dos momentos das conjunções e oposições de Júpiter, mas
garantindo que o número de ocultações ou emersões compreendidas em cada par era
o mesmo, cuidado indipensável para a validade do método.
Os
valores mais recuados no tempo estão relativamente próximos do valor correto,
mas, os calculados em anos sinódicos mais recentes estão gravemente exagerados.
\begin{table}[htb]
  \begin{center}
    \begin{tabular}{lccccc}
      \hline
      Evento &  $t_i$ (UTC) & $d_i$ (UA) & $t_f$ (UTC) & $d_f$ (UA) & $c$
      ($\times10^8$\,m/s)\\
      \hline
      emrs  & 2020-07-15 06:58:13 & 4,139337   
            & 2021-01-27 18:15:48 & 6,071236  
            & \multirow{2}{*}{2,9445}\\
      ocul  & 2021-01-29 10:26:37 & 6,071051  
            & 2021-08-13 21:10:28 & 4,019207 \\
      \hline
      ocul  & 2021-01-29 10:26:37 & 6,071051  
            & 2021-08-13 21:10:28 & 4,019207 
            & \multirow{2}{*}{2,8268}\\
      emrs  & 2021-08-21 01:21:26 & 4,139337
            & 2022-03-05 12:40:47 & 6,071235\\
      \hline
      emrs  & 2021-08-21 01:21:26 & 4,139337
            & 2022-03-05 12:40:47 & 6,071235
            & \multirow{2}{*}{3,3827}\\
      ocul  & 2022-03-07 04:53:06 & 5,971920
            & 2022-09-17 21:17:14 & 3,962823\\
      \hline
      ocul  & 2022-03-07 04:53:06 & 5,971920
            & 2022-09-17 21:17:14 & 3,962823
            & \multirow{2}{*}{3,4411}\\
      emrs  & 2022-09-26 19:53:37 & 3,952629
            & 2023-04-11 07:12:38 & 5,954576\\
      \hline
      emrs  & 2022-09-26 19:53:37 & 3,952629
            & 2023-04-11 07:12:38 & 5,954576
            & \multirow{2}{*}{4,1659}\\
      ocul  & 2023-04-12 23:29:22 & 5,955120
            & 2023-10-26 10:24:38 & 3,988829\\
      \hline
      ocul  & 2023-04-12 23:29:22 & 5,955120
            & 2023-10-26 10:24:38 & 3,988829
            & \multirow{2}{*}{4,7501}\\
      emrs  & 2023-11-04 08:58:06 & 3,983335
            & 2024-05-18 20:14:25 & 6,027377\\
      \hline
      emrs  & 2024-12-08 09:01:31 & 4,089971
            & 2025-06-22 20:11:59 & 6,158122
            & \multirow{2}{*}{3,8818}\\
      ocul  & 2025-06-26 06:55:04 & 6.159594
            & 2026-01-08 17:39:52 & 4.231754\\
      \hline
    \end{tabular}
  \end{center}
  \caption{\label{tab:bb} Valores da velocidade da luz obtidos com o método simples,
  considerando quartetos formados por pares de ocultações e de emersões próximas
  das conjunções e oposições de Júpiter com o mesmo número de revoluções de Io
  ($n=111$) em cada par.}
\end{table}



\subsection{Regressão com pares}
Aplicou-se o método de regressão com pares com os pares de eventos formados com
o evento inicial e com cada um dos restantes, para as séries de emersões e de
ocultações no período sinódico de 2024-12-07 (oposição) até (2026-01-10). Os
resultados estão apresentados nos gráficos da Figura~\ref{fig:aa}. Os valores de
$c$ resultantes são $4,564\times10^8$\,m/s e $4,307\times10^8$\,m/s,
repetivamente para a série de pares de emersões e para a de pares de ocultações.
\begin{figure}[htb]
  \begin{center}
    \includegraphics[width=0.9\textwidth]{c_rpairs.png}
  \end{center}
  \caption{\label{fig:aa} Resultados do método da regressão com pares. Gráficos
    de $\delta d/n$ (em UA) e de $\delta t/n$ (em s), para a série de emersões
    (à esquerda) e de ocultações (direita) no período 2024-12-07 a 2026-01-10.
    Os pares usados no cálculo foram todos formados com o evento inicial de cada
    série. Os valores resultantes para a velocidade da luz são
  $4,564\times10^8$\,m/s (emersões) e $4,307\times10^8$\,m/s (ocultações).}
\end{figure}
\subsection{Regressão com quartetos}
A Figura~\ref{fig:bb} mostra o gráfico obtido considerando quartetos do período
2024-12-07 a 2026-01-10. Os pares de cada quarteto têm amplitudes variando entre
10 e 97 revoluções de Io, todos centrados nas quadraturas das respetivas fases
(regressão ou aproximação).
\begin{figure}[htb]
  \begin{center}
    \includegraphics[width=0.7\textwidth]{c_rquart.png}
  \end{center}
  \caption{\label{fig:bb}Resultado do método da regressão com quartetos. Para
  cada ponto, são usadas duas emersões e duas ocultações, definindo intervalos
  temporais centrados na quadratura da fase respetiva, com amplitudes variáveis
  entre a duraação de 10 a 97 revoluções de Io. O valor resultante para $c$ é
$3,928\times10^8$\,m/s.}
\end{figure}
O valor da velocidade da luz obtido a partir do declive da reta de ajuste é
$3,928\times10^8\,$m/s.

\section{Questões}
\begin{enumerate}
  \item A que se deve a inexatidão destes resultados? A erros do stellarium? Ou
    a algum efeito ou correção observacional que não estou a considerar?
  \item É relevante que o método batota! seja bastante mais exato que os
    restantes?
  \item Como usar tabelas de efemérides (como as geradas com o sistema Horyzons
    da NASA e JPL) para determinar os instantes das emersões e ocultaações de
    Io?  (Ou deverei antes dizer das \emph{observações} desses efeitos?)
\end{enumerate}

\section{Período sinódico de Io no stellarium}
\end{document}


